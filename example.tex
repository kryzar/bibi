\documentclass[handout]{bibi}

% These packages are used in this document but not required by the class:
\usepackage{array}
\usepackage{makecell}
\usepackage{mathtools}
\usepackage{pgfplots}
\usepackage{polyglossia}

% Settings:
\usetikzlibrary{patterns}
\setmainlanguage{english}

% Colors for the intro:
\definecolor{c5}{HTML}{390099}
\definecolor{c4}{HTML}{9e0059}
\definecolor{c3}{HTML}{ff0054}
\definecolor{c2}{HTML}{ff5400}
\definecolor{c1}{HTML}{ffbd00}

% Colors for elliptic curves:
\definecolor{grad1}{HTML}{00b4d8}
\definecolor{grad2}{HTML}{00b4d8}
\definecolor{grad3}{HTML}{00b4d8}

% Colors for the comparison plot:
\definecolor{comparegreen} {HTML}{ffa69e}
\definecolor{compareblue}  {HTML}{faf3dd}
\definecolor{compareyellow}{HTML}{b8f2e6}
\definecolor{comparered}   {HTML}{aed9e0}
% As well as macros for it:
\newcommand{\FMFF}{\texttt{F-MFF}}
\newcommand{\FMKU}{\texttt{F-MKU}}
\newcommand{\FCSA}{\texttt{F-CSA}}
\DeclareMathOperator{\SM}{SM}

% \pause and \vfill at the same time:
\newcommand{\vpause}{\vfill\pause}

% Macros for displaying function definitions:
\newcommand{\Function}[5]{
  \[
    \begin{array}{rrcl}
    #1: & #2 & \to     & #3 \\
        & #4 & \mapsto & #5
    \end{array}
  \]
}
\newcommand{\FunctionNoname}[4]{
  \[
    \begin{array}{rcl}
      #1 & \to     & #2 \\
      #3 & \mapsto & #4
    \end{array}
  \]
}

% A command for displaying the table of contents during the talk:
\newcommand{\frametoc}{
  \begin{frame}
    \centering
    \Large
    \tableofcontents
  \end{frame}
}

% Math macros:
\newcommand{\Ktau}{{K\{\tau\}}}
\newcommand{\M}{{\mathbb M}}
\newcommand{\Fq}{{\mathbb{F}_q}}
\newcommand{\Kbar}{{\overline{K}}}
\newcommand{\NN}{\mathbb{Z}_{\geqslant 0}}
\newcommand{\ZZ}{\mathbb{Z}}
\newcommand{\QQ}{\mathbb{Q}}
\newcommand{\RR}{\mathbb{R}}
\newcommand{\CC}{\mathbb{C}}
% Operators
\renewcommand{\geq}{\geqslant}
\renewcommand{\leq}{\leqslant}

% Title page

\title{From elliptic curves to Drinfeld modules}

\date{
  INRIA Grace seminar \\
  \footnotesize 
  December 2nd 2025
  \normalsize
}

\author{Antoine Leudière}

\institute{University of Calgary}

\begin{document}

\begin{frame}[plain]
  \titlepage
\end{frame}

\begin{frame}{Today}

  What are \textcolor{c1}{\textbf{Drinfeld modules}}?
  How do they compare to \textcolor{c2}{\textbf{elliptic curves}}?

  \vspace{1.5em}

  How \textcolor{c3}{\textbf{effective}} are Drinfeld modules?
  \textcolor{c4}{\textbf{Counting points}} using Anderson
  motives.

  \vspace{1.5em}

  Potential \textcolor{c5}{\textbf{applications}}.

  \vspace{2em}

  \begin{flushright}

    Joint work with Xavier Caruso.

    \emph{Algorithms for computing norms and characteristic polynomials on
    general Drinfeld modules}. Mathematics of Computation. 2026.

  \end{flushright}

\end{frame}

\frametoc

\section{The rules of point counting}

\begin{frame}{What is point counting?}

  \begin{block}{Naively}

    Counting solutions to an equation.

  \end{block}

  \vpause

  \begin{block}{Generally a hard problem}

  \begin{itemize}

    \item Algebraic varieties on a finite field.
    \item Matiyasevich's theorem (1970): no algorithm can tell if any given
      Diophantine equation has integer solutions.

  \end{itemize}

  \end{block}

  \vpause

  Consider geometric objects with more structure: \emph{elliptic curves}.

\end{frame}

\begin{frame}{Elliptic curves}
  \begin{columns}
    \begin{column}{0.4\textwidth}
      \begin{tikzpicture}[scale=0.7]

        \begin{axis}[
            axis lines=middle,
            axis line style={gray},
            xtick=\empty,
            ytick=\empty,
            xlabel style={anchor=north, gray, font=\footnotesize},
            ylabel style={anchor=east,  gray, font=\footnotesize},
            xmin=-3,
            xmax=+4,
            ymin=-4,
            ymax=+4,
            scale only axis,
            axis equal,
        ]

        % Plot the elliptic curve
        \addplot[on layer=main, domain=-2.38297:3, smooth, line width=0.8pt, samples=100] {sqrt(x^3-4*x+4)};
        \addplot[on layer=main, domain=-2.38297:3, smooth, line width=0.8pt, samples=100] {-sqrt(x^3-4*x+4)};
  
        % Add points
        \addplot[only marks, mark=*, mark options={draw=grad1, fill=grad2}] coordinates {(1,1)};
        \addplot[only marks, mark=*, mark options={draw=grad2, fill=grad2}] coordinates {(-1.75,-2.375)};
        \addplot[only marks, mark=*, mark options={draw=grad3, fill=grad2}] coordinates {(2.2562,-2.5417)};
  
        % Label the points
        \node[above, font=\footnotesize] at (axis cs:1,1)               {$P$};
        \node[below, font=\footnotesize] at (axis cs:-1.75,-2.375)      {$Q$};
        \node[below, font=\footnotesize] at (axis cs:2.2562,-2.5417)    {$P+Q$};
  
        \addplot[domain=-4:4, line width=0.6pt, red, dotted] {{1.2273*x - 0.2273}};
        \addplot[domain=-4:4, line width=0.6pt, red, dotted] coordinates {(2.2562,4) (2.2562,-4)};
  
        \end{axis}
      \end{tikzpicture}
    \end{column}

    \begin{column}{0.55\textwidth}

      Smooth algebraic projective curves of genus $1$ with a distinguished
      point.

      \begin{exampleblock}{Double nature}
        Arithmetic-geometric objects.
      \end{exampleblock}

      \vspace{2em}
      \pause

      \begin{exampleblock}{Applications}
        \begin{itemize}
          \item Number theory
          \pause
          \item Cryptography (pre \& post-quantum)
          \item Computer algebra (ECPP, ECM)
        \end{itemize}
      \end{exampleblock}

    \end{column}
  \end{columns}
\end{frame}

\begin{frame}{Changing the rules}

  \begin{exampleblock}{}

    Let $E$ be an elliptic curve over $\Fq$. As an abelian group,
    \[
      E(\Fq) \simeq \ZZ/d_1\ZZ \times \cdots \times \ZZ/d_n \ZZ.
    \]
    So
    \[
      \# E(\Fq) = |d_1 \cdots d_n|.
    \]

  \end{exampleblock}

  \vpause

  Let $R$ be a PID, $M$ be a finite $R$-module.
  There are $m_1, \dots, m_\ell \in R$ s.t.:
  \[
    M \simeq R/m_1 R \times \cdots \times R/m_\ell R.
  \]

  \pause

  \begin{block}{$R$-cardinality}

    Define the \emph{$R$-cardinality of $M$} as
    \[
      m_1 \cdots m_\ell.
    \]

  \end{block}

\end{frame}

\begin{frame}{Drinfeld modules!}

  \begin{alertblock}{}
    
    Replace $\ZZ$ by $R = \Fq[T]$!

  \end{alertblock}

  \vpause

  Both Euclidean domains.

  \begin{block}{Analogies}
  \begin{center}
    \begin{tabular}{r|l} \hline
      $\ZZ$
      & $\Fq[T]$
      \\ \hline
      \hline

      \pause

      $\QQ$
      & $\Fq(T)$
      \\ \hline

      \makecell[r]{Number fields \\ (finite extensions of $\QQ$)}
      &
      \makecell[l]{Function fields \\ (finite extensions of $\Fq(T)$)}
      \\
      \hline

      \pause

      $\RR$
      & $\RR_\infty = \Fq((\frac 1 T))$
      \\ \hline

      $\CC$
      & $\CC_\infty = $ completion of $\overline{\RR_\infty}$
      \\ \hline
      \hline

      \pause

      \textbf{Elliptic curves} & \textbf{Drinfeld modules}
      \\ \hline
    \end{tabular}
  \end{center}
  \end{block}

  \vpause

  \begin{alertblock}{Mantra}

    Our integers are polynomials.

  \end{alertblock}


\end{frame}

\section{A new area}

\begin{frame}{From elliptic curves to Drinfeld modules}

  \begin{center}
    \begin{tabular}{r|l|l}
       & Elliptic curves & Drinfeld modules \\
      \hline
      \hline
      Introduction & 1850-1900 & 1977 \\
      \hline
      Practical applications & 1980s & \textcolor{c4}{2021}
    \end{tabular}
  \end{center}

  \vpause

  \begin{block}{}
  
    Drinfeld modules were introduced (and were successful) for:
    \begin{itemize}

      \item Class field theory (Kronecker-Weber, complex multiplication).

      \item Langlands conjectures for function fields ($\mathrm{GL}_2$ then
        $\mathrm{GL}_r$).

    \end{itemize}

  \end{block}

  \vpause
  
  \begin{alertblock}{}
  
    Research on algorithmics of Drinfeld modules is a very new area!
  
  \end{alertblock}

  \vpause 

  \begin{block}{Our goal}

    \begin{itemize}
      
      \item Modern techniques for manipulating Drinfeld modules.

      \item Efficiency and generality (rank and function fields).
    
      \item Applications (coding theory, computer algebra).

    \end{itemize}

  \end{block}

\end{frame}

\begin{frame}{Timeline}

  \begin{itemize}

    \item Early works on computational aspects: \textcolor{c2}{1980s} (Gekeler,
      Bae \& Koo, etc).

    \item First thesis on the computational aspects: \textcolor{c3}{2018} (Caranay).

    \item First computer algebra application: \textcolor{c4}{2021} (Doliskani,
      Narayanan, Schost).

    \item First high generality algorithms:
      \textcolor{c5}{2023} (Musleh \& Schost, Caruso \& Leudière).

  \end{itemize}

  \vpause

  \begin{block}{My research}

    \begin{itemize}

      \item Computer algebra of Drinfeld modules (Caruso-L. 2023, L. 2026).
      \item SageMath implementation (Ayotte-Caruso-L.-Musleh 2023).
      \item Algorithmics of function fields (L.-Spaenlehauer, 2023).
      \item (Small cyclotomic integers (Bajpai, Das, Kedlaya, Le, L. Lee, Mello,
        2025).)

    \end{itemize}

  \end{block}

\end{frame}

\section{Representing Drinfeld modules}

\begin{frame}{Ore polynomials}

  Fix $K/\Fq$, and for all $n \in \ZZ_{\geq 0}$:
  \Function
    {\tau^n}
    {\Kbar}
    {\Kbar}
    {x}
    {x^{q^n}.}

  \begin{block}{Definition (Ore polynomials)}
    
    $\Ktau = $ finite $K$-linear combinations of $\tau^n$'. Ring for addition
    and composition.

  \end{block}

  \vpause

  \begin{block}{Properties}

    \begin{itemize}

      \item Representation as polynomials: $\Ktau = \{\sum_{i=0}^n x_i \tau^i,
        n\in \NN, x_i \in K\}$.

      \pause

      \item Notion of $\tau$-degree.

      \pause

      \item Noncommutative: for $\lambda \in K$, $\tau^n \lambda =
        \lambda^{q^n} \tau^n$.

      \pause

      \item Left-euclidean: for any $A, B \in \Ktau$, there exist $Q, R \in
        \Ktau$ such that:
       \[
           A = QB + R, \quad \deg_\tau(R) < \deg_\tau(B).
       \]

    \end{itemize}

  \end{block}

\end{frame}

\begin{frame}{Representing Drinfeld modules}

  \begin{block}{(Almost) Definition (Drinfeld, 1977)}

    A \emph{Drinfeld $\Fq[T]$-module over $K$} is a homomorphism
    of $\Fq$-algebras
    \Function
      {\phi}
      {\Fq[T]}
      {\Ktau}
      {a}
      {\phi_a.}

  \end{block}

  \vpause

  \begin{block}{Morphisms}
  
    A \emph{morphism} $u: \phi \to \psi$ is an Ore polynomial $u \in \Ktau$ such
    that
    \[
      \forall a \in \Fq[T], \qquad u \phi_a = \psi_a u.
    \]
    An \emph{isogeny} is a nonzero morphism.
  
  \end{block}

\end{frame}

\begin{frame}{The rank of a Drinfeld module}

  \begin{block}{Definition (rank)}

    $\phi$ is represented by $\phi_T$. The \emph{rank} of $\phi$ is
    $\deg_\tau(\phi_T)$.

  \end{block}

  \vpause

  \begin{alertblock}{}

    Elliptic curves correspond to rank $2$ only! (Lattices in $\CC$ vs
    $\CC_\infty$.)

  \end{alertblock}

  \vpause

  \begin{block}{Point counting state of the art}

  \begin{tabular}{rll}
    
    \textcolor{gray}{\textit{2008}} & Gekeler         & Frobenius, $r = 2$ generalized to $r \in \NN$ by Musleh \\
    \textcolor{gray}{\textit{2019}} & Musleh, Schost  & Frobenius, $r = 2$  \\
    \textcolor{gray}{\textit{2020}} & Garai, Papikian & Frobenius, $r = 2$  \\
    \textcolor{gray}{\textit{2023}} & Musleh, Schost  & Any endomorphism, any $r$ \\
    \textcolor{gray}{\textit{2023}} & \textbf{Caruso, L.}  & Any endomorphism,
    any $r$, field, function field  \\
    & & + isogeny norms \\

  \end{tabular}
  \end{block}

\end{frame}

\begin{frame}{The points of a Drinfeld module}

  \begin{exampleblock}{}
    For an elliptic curve, the \emph{points} form a $\ZZ$-module.
  \end{exampleblock}
  
  \vpause

  \begin{block}{Geometric points}

    The \emph{$\Fq[T]$-module of points}, denoted by $\phi(\Kbar)$, is given
    by:
    \FunctionNoname
      {\Fq[T] \times \Kbar}
      {\Kbar}
      {(a, z)}
      {\phi_a(z).}

  \end{block}

  \pause

  \begin{block}{$K$-rational points}

    The \emph{$\Fq[T]$-module of $K$-rational points} is
    \[
      \phi(K) \coloneqq \phi(\Kbar) \cap K.
    \]

  \end{block}

  \vpause

  \begin{alertblock}{}
    The underlying set of $\phi(K)$ is always $K$!
  \end{alertblock}

\end{frame}

\begin{frame}{The number of points}

  \begin{exampleblock}{}

    For an elliptic curve,
    \[
      E(\Fq) \simeq \ZZ/(d_1) \times \cdots \times \ZZ/(d_n),
    \]
    and
    \[
      (\# E(\Fq)) \simeq (d_1 \cdots d_n)
    \]

  \end{exampleblock}

  \vpause

  \begin{block}{}

    Assume $K$ is finite. Decompose
    \[
      \phi(K) \simeq \Fq[T]/(d_1) \times \cdots \times \Fq[T]/(d_n).
    \]
    The ``number of $K$-rational points of $\phi$'' ($\Fq[T]$-cardinality) is
    \[
      (|\phi(K)|) = (d_1 \cdots d_n).
    \]

  \end{block}

  Often referred to as the \emph{Euler-Poincaré characteristic} or
  \emph{Fitting ideal} of $\phi(K)$.

\end{frame}

\section{Point counting without points}

\begin{frame}{The elliptic curve case}

  First deterministic polynomial time: Schoof, 1985.

  \pause

  \begin{exampleblock}{Number of points \emph{via} the Frobenius endomorphism}

    \begin{enumerate}

      \item An elliptic curve $E/\Fq$ has a \emph{Frobenius endomorphism} $F:
        (x, y) \mapsto (x^q, y^q)$.

      \pause

      \item $F$ has a \emph{characteristic polynomial}
        \[
          \chi = X^2 - tX + q \in \ZZ[X]
        \]
        such that
        \[
          \chi(F) = F^2 - tF + q = 0.
        \]

      \pause

      \item We have
        \[
          |E(\Fq)| = \chi(1).
        \]

    \end{enumerate}

    \vpause
  
  \end{exampleblock}

  Important invariant.

\end{frame}

\begin{frame}{The Drinfeld module case}

  \begin{block}{}
  \begin{enumerate}

    \item Assume $K$ is finite. A Drinfeld module $\phi$ over $K$ has a
      \emph{Frobenius endomorphism} $F = \tau^{[K: \Fq]} \in \Ktau$.

    \pause

    \item $F$ has a \emph{characteristic polynomial}
    \[
      \chi = X^r + a_{r-1}(T) X^{r-1} + \cdots + a_1(T)X + a_0(T) \in \Fq[T][X]
    \]
    such that
    \[
      \chi(F) = F^r + \phi_{a_{r-1}} F^{r-1} + \cdots +
      \phi_{a_1} F + \phi_{a_0} = 0.
    \]

    \pause

    \item We have (Gekeler, 1991)
    \[
      (|\phi(K)|) = (\chi(1))
    \]

  \end{enumerate}

  \end{block}

    \vpause

  Important invariant.

\end{frame}

\begin{frame}{Abstract definition of $\chi$}

  \begin{block}{Via \emph{Tate modules}}
  \begin{enumerate}
        
    \item Make $\Fq[T]$ act on $\Kbar$ \emph{via} $\phi$.

    \item Consider the action of $F$ on (almost all) the $\ell$-torsion
      submodules, $\ell \in \Fq[T]$.

    \item Show that these are free with rank $r$ on  $\Fq[T]/(\ell)$.

    \item Show that the characteristic polynomial of the action of $F$ on
      these modules lifts to a single polynomial $\chi \in \Fq[T][X]$.

  \end{enumerate}
  \end{block}

  \vpause

  \begin{block}{Problem}

    \begin{itemize}
      \item Manipulate torsion elements in possibly large extensions.
      \item Or derive an efficient theory of \emph{division polynomials}.
    \end{itemize}

  \end{block}

\end{frame}

\begin{frame}{Anderson motives}

  \begin{block}{Definition (Anderson motive of $\phi$)}

    $\M(\phi)$ is the $K[T]$-module
    \FunctionNoname
      {K[T] \times \Ktau}
      {\Ktau}
      {\left(\sum_i \lambda_i T^i, f(\tau)\right)}
      {\sum_i \lambda_i f(\tau) \phi_T^i}

  \end{block}

  \vpause

  \begin{block}{Canonical basis}
    $\mathbb M(\phi)$ is free with rank $r$ (the rank of $\phi$) with basis
    \[(1, \tau, \dots, \tau^{r-1}).\]
  \end{block}

  \vpause

  \begin{alertblock}{Explicit decomposition in the canonical basis}
    Ore Euclidean division and recursion:
    \[
      f(\tau) = Q(\tau) \phi_T + R(\tau), \quad \deg_\tau(R) < r =
      \deg_\tau(\phi_T).
    \]
  \end{alertblock}

\end{frame}

\begin{frame}{Morphisms as matrices}

  \begin{block}{}

    Any morphisms $u: \phi \to \psi$ gives a morphism on the Anderson motives
    \Function
      {\M(u)}
      {\M(\psi)}
      {\M(\phi)}
      {f}
      {fu.}

  \end{block}

  \vpause

  \begin{alertblock}{Effective computation}
    To compute the matrix of $\M(u)$, compute the coordinates of
    \[
      u, \tau u, \cdots, \tau^{r-1} u.
    \]
  \end{alertblock}

\end{frame}

\begin{frame}{Norms and characteristic polynomials}

  Let $u: \phi \to \psi$ be an isogeny of Drinfeld modules.

  Consider $\M(u):\M(\psi)\to\M(\phi)$ as a matrix in the canonical bases.

  \begin{block}{}

    \begin{itemize}

    \item If $u$ is an endomorphism, its characteristic polynomial is that of
    $\M(u)$.

    \item The \emph{norm} of $u$ is $\det(\M(u))$.
    \end{itemize}

  \end{block}

  \vpause

  \begin{block}{Our work}

    \begin{itemize}
    
      \item Prove it (for any function ring, field, rank, isogeny).

      \item Multiple variants, optimization, analysis.

      \item Implementation.

      \item (An extra algorithm, only for the Frobenius, based on reduced
        norms.)

    \end{itemize}

  \end{block}

\end{frame}

\begin{frame}{}

  % Thank you Xavier Caruso for this plot!

  \def\cA{(0.3805, 0.761)}
  \def\cB{(0.3805, 0)}
  \def\cC{(0.4308, 1)}
  \def\ph{\vphantom{$A^A_A$}}
  
  \centering
  \hfill%
  \begin{tikzpicture}[scale=6.6]
  \fill[pattern=north east lines] (0,1) rectangle (1.43,1.08);
  \fill[comparegreen] (0,0)--\cA--\cC--(0,1);        % musleh-schost
  \fill[comparered] (0,0)--\cA--\cB;                 % F-MKU
  \fill[compareyellow] \cB--\cA--\cC--(1,1)--(1,0);  % F-MFF
  \fill[compareblue] (1,0) rectangle (1.43,1);       % F-CSA
  \begin{scope}[thick]
  \draw[-latex] (0,0)--(1.45,0);
  \draw[-latex] (0,0)--(0,1.13);
  \draw (0,1)--(1.43,1);
  \draw (0,0)--\cA--\cC;
  \draw \cA--\cB;
  \draw (1,0)--(1,1);
  \end{scope}
  \node[scale=0.8] at (1.47,-0.04) { $\frac{\log r}{\log d}$ };
  \node[scale=0.8] at (-0.035, 1.15) { $\frac{\log m}{\log d}$ };
  \draw[dotted] \cA--(0, 0.761);
  \draw[dotted] \cC--(0.4308, 0);
  \node[scale=0.8] at (-0.01,-0.02) { $0$ };
  \node[scale=0.8] at (-0.02,1) { $1$ };
  \node[scale=0.8] at (1,-0.03) { $1$ };
  \node[scale=0.8] at (0.37,-0.04) { \hspace{-0.6cm} $\frac 1{5-\omega}$ };
  \node[scale=0.8] at (-0.04, 0.761) { $\frac 2{5-\omega}$ };
  \node[scale=0.8] at (0.46,-0.04) { \hspace{0.2cm} $\frac {\omega+3\hspace{3mm}}{10\omega - 2 \omega^2}$ };
  \node at (0.17,0.7) { \ph \scriptsize Musleh-Schost };
  \node at (0.26,0.25) { \ph \scriptsize \FMKU };
  \node at (0.68,0.5) { \ph \scriptsize \FMFF };
  \node at (1.22,0.5) { \ph \scriptsize \FCSA };
  \end{tikzpicture}%
  \hfill\null

\end{frame}

\begin{frame}{Conclusion}

  \begin{block}{}
  Problems inspired from elliptic curves.

  New solutions (efficiency, generality).

  \end{block}

  \vpause

  \begin{block}{Potential of Drinfeld modules}
    
    \begin{itemize}

      \item Reveal differences between number fields and function fields.

      \item Computer algebra of polynomials.

      \item Coding theory:
        \begin{itemize}
          \item Drinfeld modular curve (asymptotically good towers of
            curves).
          \item Function Field Decoding Problem (Bombar,
            Couvreur \& Debris-Alazard).
          \item Rank-metric, locally recoverable codes (Bastioni, Darwish \&
            Micheli).
        \end{itemize}

    \end{itemize}

  \end{block}

\end{frame}

\end{document}
